\documentclass[conference]{IEEEtran}
\usepackage{graphicx}
\usepackage{array}

\title{How AI is Changing Healthcare: A Casual Look}

\author{
\begin{center}
\begin{tabular}{>{\centering\arraybackslash}p{0.3\linewidth} >{\centering\arraybackslash}p{0.3\linewidth} >{\centering\arraybackslash}p{0.3\linewidth}}

  
    
      \begin{minipage}[t]{\linewidth}
        \centering
        {\bfseries krish}\\
        Srilanka insitute of information technology\\
        shandeep@gmail.com
      \end{minipage}
       & 
    
      \begin{minipage}[t]{\linewidth}
        \centering
        {\bfseries sanjayan}\\
        Srilanka insitute of information technology\\
        sanjayan@gmail.com
      \end{minipage}
       & 
    
      \begin{minipage}[t]{\linewidth}
        \centering
        {\bfseries krith}\\
        Srilanka insitute of information technology\\
        krithy@gmail.com
      \end{minipage}
      
    
     \\
  

\end{tabular}
\end{center}
}

\begin{document}
\maketitle

\begin{abstract}
This paper takes a relaxed look at how artificial intelligence (AI) is starting to change the way healthcare works. In everyday terms, AI refers to computer systems that can perform tasks that usually require human intelligence, and it’s being used in a lot of cool ways—from predicting patient outcomes to automating routine tasks. Machine learning (ML), a key subset of AI, is especially hot right now and is helping doctors and hospitals make smarter decisions. For example, many clinics are now using electronic health records (EHR) powered by ML algorithms to spot potential issues before they become serious. The paper doesn’t dive too deep into the technical jargon but instead explains the basic ideas behind these technologies and how they’re applied in real-world settings. It also covers some of the challenges like data privacy, the need for good quality data, and the gap between technical innovation and everyday use. Overall, this informal overview aims to give readers a taste of the exciting changes happening in healthcare due to AI, while acknowledging that there’s still a long way to go before these systems become the norm.
\end{abstract}

\begin{IEEEkeywords}
artificial intelligence, AI, machine learning, ML, healthcare, electronic health records, EHR
\end{IEEEkeywords}




  \section{Introduction}
  Artificial intelligence (AI), which refers to computer systems that can learn and make decisions, is revolutionizing healthcare through its applications in diagnosis, patient management, and data analysis. Machine learning (ML), a subset of AI that uses algorithms to improve prediction accuracy, has become an essential tool in healthcare informatics, enabling the analysis of large electronic health record (EHR) datasets to identify potential health issues before they become emergencies. While the adoption of AI in healthcare presents exciting opportunities for improved patient outcomes, concerns about the potential for bias and ethical considerations remain. This paper provides a comprehensive overview of the current state of AI in healthcare, highlighting its benefits, challenges, and real-world applications.

\begin{figure}[h]
\centering
\includegraphics[width=0.5\textwidth]{..//uploads/1740632573014.jpg}
\caption{universe}
\end{figure}
  



  \section{Literature Review}
  The prevalence of informal discourse on artificial intelligence (AI) in healthcare is evident from the vast amount of online content, including tech blogs, social media posts, and academic journals. These sources highlight the potential of AI, particularly machine learning (ML), to analyze electronic health records (EHR) and other patient data. However, the literature is inconsistent, with some articles touting the transformative impact of AI on diagnostics and treatment planning, while others caution against its limitations due to data quality, privacy concerns, and the steep learning curve for healthcare professionals. The informal tone of much of this literature reflects the enthusiasm and excitement surrounding AI in healthcare, yet acknowledges its early stage of development and the regulatory and ethical challenges that remain.





  \section{Results / Discussion}
  The uptake of electric vehicles (EVs) has increased significantly, as evidenced by the data on their sales. Drivers appreciate the cost savings associated with EVs, including reduced fuel and maintenance expenses. However, the lack of public charging infrastructure remains a major concern, limiting their adoption. Addressing this gap is crucial to ensuring the widespread adoption of EVs.






  

  

  
    \section{Methedology}
    This paper adopted an informal, exploratory approach to gather insights into the application of artificial intelligence (AI) in healthcare. By scouring the internet for blog posts, forum discussions, and online reviews, the paper identified key trends and challenges associated with the use of machine learning (ML) for processing electronic health records (EHR) and improving patient outcomes. While the study did not employ advanced data collection tools, it provided valuable insights into the real-world experiences and perspectives of healthcare professionals. However, the paper acknowledged the limitations of its unstructured methodology, highlighting the need for further research to address technical details and mitigate biases in the analysis.

\begin{figure}[h]
\centering
\includegraphics[width=0.5\textwidth]{..//uploads/1740581299422.jpg}
\caption{1st image}
\end{figure}
    
  

  

  
    \section{Conclusion (and possibly Future Work)}
    The informal investigation into artificial intelligence (AI) and machine learning (ML) in healthcare demonstrates a mixed sentiment. On the one hand, the use of AI and ML in analyzing electronic health records (EHR) has shown promising potential for improving patient care and streamlining hospital operations. On the other hand, many professionals remain apprehensive due to concerns regarding data quality, privacy, and the complexity of integrating these new systems into traditional healthcare settings. Despite the enthusiasm surrounding AI, formal research studies are necessary to validate its benefits and address its challenges. This paper provides an informal perspective on the current state of AI in healthcare, highlighting both its potential and the obstacles that must be overcome for these technologies to become widespread.

\begin{figure}[h]
\centering
\includegraphics[width=0.5\textwidth]{..//uploads/1740581318803.jpg}
\caption{image 2}
\end{figure}
    
  




\begin{thebibliography}{1}

  
    \bibitem{ref-1740062965592}
    
  

\end{thebibliography}

\end{document}
